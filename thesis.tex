\documentclass[12pt,oneside]{fithesis2}
\usepackage[english]{babel}       % Multilingual support
\usepackage[utf8]{inputenc}       % UTF-8 encoding
\usepackage[T1]{fontenc}          % T1 font encoding
\usepackage[                      % A sans serif font that blends well with Palatino
  scaled=0.86
]{berasans}
\usepackage[                      % A tt font if you do not like LM's tt
  scaled=1.03
]{inconsolata}
\usepackage[                      % Clickable links
  plainpages = false,               % We have multiple page numberings
  pdfpagelabels                     % Generate pdf page labels
]{hyperref}
\usepackage{blindtext}
\usepackage{url}            % Lorem ipsum generator

\thesislang{en}                   % The language of the thesis
\thesistitle{Extend WildFly Application Server logging capabilities}       % The title of the thesis
\thesissubtitle{Diploma Thesis}  % The type of the thesis
\thesisstudent{Bc. Radek Koubský}          % Your name
\thesiswoman{true}                % Your gender
\thesisfaculty{fi}                % Your faculty
\thesisyear{\the\year}     % The academic term of your thesis defense
\thesisadvisor{RNDr. Adam Rambousek, Ph.D.}   % Your advisor

\begin{document}
  \FrontMatter                    % The front matter
    \ThesisTitlePage                % The title page
    \begin{ThesisDeclaration}       % The declaration
      \DeclarationText
      \AdvisorName
    \end{ThesisDeclaration}
    \begin{ThesisThanks}            % The acknowledgements (optional)
      I would like to thank my supervisor\,\dots
    \end{ThesisThanks}
    \begin{ThesisAbstract}          % The abstract
      The aim of the bachelor work is to provide\,\dots
    \end{ThesisAbstract}
    \begin{ThesisKeyWords}          % The keywords
      keyword1, keyword2\,\dots
    \end{ThesisKeyWords}
    \tableofcontents                % The table of contents
%   \listoftables                   % The list of tables (optional)
%   \listoffigures                  % The list of figures (optional)
  
  \MainMatter                     % The main matter
    \chapter{Introduction}          % Chapters
    \Blindtext
\chapter{Byteman}
This chapter aims to introduce \textit{Byteman}, a bytecode manipulation tool, and its usage in java applications. Fundamental principles, functions and structure of Byteman will be described.
At the end of the chapter, an example of simple java application will be provided in order to understand bytecode injection which Byteman uses. More detailed documentation, tutorials and guidelines 
which cover almost every aspect of Byteman can be found in different versions at official website \url{http://byteman.jboss.org/}. Primary source for this thesis is Byteman documentation 3.0.0 \cite{byteman_doc}.

\section{Introduction to Byteman}
Byteman is a bytecode manipulation tool which uses a bytecode injection\footnote{Bytecode injection (or bytecode instrumentation) is a process of modifying the actual bytecode of a class} in order to change a java application either as it is being loaded by \textit{Java virtula machine} or while the application is running \cite[Introduction to Byteman]{byteman_doc}.

Because of the bytecode injection, there is no need to change the original source code nor to recompile, redeploy the application itself. As it was mentioned, it is possible to add, replace or remove injected code at runtime while the application continues in execution. Byteman also provides a way to inject code into the Java language by modifying classes such as \textit{java.lang.Thread} and other classes from \textit{java.lang} package. Byteman uses simple scripting language called Event Condition Action\footnote{Event Condition Action (ECA) languages are an intuitive and powerful paradigm for
programming reactive systems \cite{eca}} rules to determine how the original Java code should be transformed according to event and condition.

The original purpose of Byteman was to provide a mechanism for automation tests used in multi-threaded and multi-JVM Java applications using a technique called fault injection. Byteman can be used to inject faults which cause an application to invoke unusual or unexpected methods which are required for certain test scenarios. There are four areas supported by Byteman in automation testing \cite[Introduction to Byteman]{byteman_doc}:

\begin{itemize}
   \item tracing execution of specific code paths and displaying application or JVM state
   \item subverting normal execution by changing state, making unscheduled method calls or forcing an unexpected return or throw
   \item orchestrating the timing of activities performed by independent application threads
   \item monitoring and gathering statistics summarising application and JVM operation
\end{itemize}

However, the concept of bytecode injection involves Byteman in much larger scope of use apart from the testing area. The rule engine which forms the core of Byteman is able to inject bytecode into
almost every place in a call stack of a running java application. Thus the capability of bytecode injection enables manipulation of a java program without any limitations in the design of the language (e.g. access level modifiers).

Byteman offers a suite of built-in operations and conditions in order to support activities and tasks emphasized above. Built-in operations (such as delays, waits, countdowns, flag operations and signals) are mainly used for coordination of threads in multi-threaded applications where multiple threads run with no order and it is needed to control the flow of particular threads.

To simplify process of using Byteman in common usage or in the test automation, the Byteman rulecheck plugin and the BMUnit projects were created in later releases of Byteman.

Byteman rulecheck plugin\footnote{Byteman rulecheck plugin - \url{https://developer.jboss.org/wiki/CheckingYourBytemanRuleScriptsUnderMaven}} is a maven plugin which consits of a rule parser and a type checker. It runs the rule parser and type checker before every execution of any maven test to ensure that rules are still valid when a test code is modified.

Byteman is an \textit{Open Source} project and plays important role in \textit{JBoss Community}. It helps other JBoss projects with test automation and quality assurance.
    \appendix
    \chapter{First appendix}        % Appendices
    \Blindtext
    \chapter{Another appendix}
    \Blindtext

\bibliographystyle{IEEEtran}
\bibliography{sources}
\end{document}
